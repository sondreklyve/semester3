\documentclass[11pt, a4paper, norsk]{NTNUoving}
\usepackage[utf8]{inputenc}
\usepackage[T1]{fontenc}
\usepackage{mathrsfs}
\newcommand{\RomanNumeralCaps}[1]
{\MakeUppercase{\romannumeral #1}}


\ovingnr{4}    % Nummer på innlevering
\semester{Haust 2022}
\fag{TMA4120}
\institutt{Institutt for matematiske fag}
\begin{document}
\section*{11.4}
\begin{oppgave}[2]
Starter med å finne Fourierrekken til $f(x)$. Siden funksjonen er odde, blir $a_n=0$, og koeffisientene blir
\[
b_n=\frac{2}{\pi}\int_0^{\pi}x\sin(nx) dx = -\frac{2(-1)^n}{n}	
\]
Bruker videre at
\[
E^{*} = \int_{-\pi}^{\pi}f^2 dx - \pi \left[2a_0^2 + \sum_{n=1}^{N} (a_n^2+b_n^2)\right]
\]
Setter inn for koeffisientene, og får
\begin{align*}
E^{*} &= \int_{-\pi}^{\pi}x^2 dx - \sum_{n=1}^{N}\frac{4\pi}{n^2} \\
	&= \frac{2\pi^3}{3}-  \sum_{n=1}^{N}\frac{4\pi}{n^2} \\
\end{align*}
Regner ut noen verdier i python
\begin{center}
\begin{tabular}{ll}
$N$ & $E^{*}$\\
1& 8.104480505840705\\
2& 4.962887852250912\\
3& 3.5666244506554463\\
4& 2.7812262872580007\\
5& 2.278571462683633\\
10& 1.1958954441865721\\
100& 0.1250374819660962\\
1000& 0.012560089523425688 \\
\end{tabular}
\end{center}
\end{oppgave}
\begin{oppgave}[3]
Starter med å finne Fourierrekken til $f(x)$. Siden funksjonen er like, blir $b_n=0$, og koeffisientene blir
\begin{align*}
a_0 &= \frac{1}{\pi}\int_0^{\pi}x dx = \frac{\pi}{2}\\
a_n&=\frac{2}{\pi}\int_0^{\pi}x\cos(nx) dx = -\frac{2}{\pi n^2}((-1)^n-1)	
\end{align*}
Setter inn for koeffisientene, og får
\begin{align*}
E^{*} &= \int_{-\pi}^{\pi}x^2 dx - \pi \left[\frac{2\pi^2}{4} + \frac{4}{\pi^2}\sum_{n=1}^{N} \frac{((-1)^n-1)^2}{n^4}\right] \\
 &=\frac{\pi^3}{6} - \frac{4}{\pi}\sum_{n=1}^{N} \frac{((-1)^n-1)^2}{n^4} \\
\end{align*}
Regner ut noen verdier i python
\begin{center}
\begin{tabular}{ll}
$N$ & $E^{*}$\\
1& 0.07475460110931831 \\
2& 0.07475460110931831 \\
3& 0.011878574208815884 \\
4& 0.011878574208815884 \\
5& 0.0037298411225110684 \\
\end{tabular}
\end{center}
\end{oppgave}
\begin{oppgave}[9]
Finner koeffisienten
\begin{align*}
c_n &= \frac{1}{2\pi}\int_{-\pi}^{\pi}x e^{-inx} dx \\
&= \frac{ix}{2\pi n}\left[e^{-inx}\right]_{-\pi}^{\pi} -
	  \frac{i}{2\pi n} \int_{-\pi}^{\pi}e^{-inx} dx \\
&= \frac{i}{n} \cos(n\pi)+ \frac{1}{2\pi n^2} \left[e^{-inx} \right]_{-\pi}^{\pi} \\
&= \frac{i}{n}\cos(n\pi)- \frac{i}{\pi n^2} \sin(n\pi) \\
&= i \frac{(-1)^n}{n}
\end{align*}
Fra dette kan vi skrive ut Fourierrekken som
\[
f(x) = i\sum_{n=-\infty}^{\infty}\frac{(-1)^n}{n}e^{inx}
\]
\end{oppgave}
\begin{oppgave}[13]
Tar utgangspunkt i funksjonen 
\[
f(x) =
\left\{
	\begin{array}{ll}
		\pi - x  & \mbox{if } x > 0 \\
		\pi + x & \mbox{if } x < 0
	\end{array}
\right.
\]
Dette er en like funksjon, så Fourierkoeffisientene blir
\begin{align*}
a_0 &= \frac{1}{\pi} \int_0^{\pi} (\pi-x) dx = \frac{\pi}{2}\\
a_n &= \frac{2}{\pi} \int_0^{\pi} (\pi-x)\cos(nx) dx = \frac{2-2(-1)^n}{\pi n^2}
\end{align*}
Setter disse koeffisientene inn i Parsevals identitet, som gir
\begin{align*} 
&& 2a_0^2+\sum_{n=1}^{\infty} (a_n^2 + b_n^2) &= \frac{1}{\pi}\int_{-\pi}^{\pi} f(x)^2 dx \\
&\Rightarrow& \frac{\pi^2}{2}+\frac{4}{\pi^2}\sum_{n=1}^{\infty} \frac{(1-(-1)^n)^2}{n^4} &= \frac{\pi^2}{3}\\
&\Rightarrow& \frac{1}{4}\sum_{n=1}^{\infty} \frac{(1-(-1)^n)^2}{n^4} &= \frac{\pi^4}{96}\\
\end{align*}
Dersom vi skriver ut summen, ser vi at det stemmer med oppgaven.
\[
\frac{1}{4}\sum_{n=1}^{\infty} \frac{(1-(-1)^n)^2}{n^4}= 1 + \frac{1}{3^4} + \frac{1}{5^4} + \frac{1}{7^4} + ... = \frac{\pi^4}{96}
\]
\end{oppgave}
\section*{11.7}
\begin{oppgave}[1]
Begynner med å se på de enkleste tilfellene. Dersom $x=0$, får vi integralet
\[
\int_0^{\infty} \frac{1}{1+w^2} dw = \frac{\pi}{2}
\]
som stemmer med oppgaven.\\
Videre ser vi at funksjonen er definert som et integral fra $0$ til $\infty$, slik at funksjonsverdien blir $0$ for alle $x<0$.\\
Til sist har vi tilfellet der $x>0$. Vi deler opp integralet i sinus- og cosinusledd, og bruker deretter (4) og (5) fra Kreyzsig for å sjekke opp den oppgitte funksjonen stemmer.
\begin{align*}
A(w)&=\frac{1}{\pi}\int_0^{\infty}\pi e^{-v}\cos(wv)dv = \frac{1}{1+w^2}\\ 
B(w)&=\frac{1}{\pi}\int_0^{\infty}\pi e^{-v}\sin(wv)dv = \frac{w}{1+w^2}\\ 
\end{align*}
Disse integralene kjenner vi uansett løsningene på fra Laplacetransformasjoner. Vi ser at svarene stemmer overens med $A(w)$ og $B(w)$ fra integralet, og kan dermed konkludere at likheten i oppgaven er korrekt.
\end{oppgave}
\section*{11.9}
I denne seksjonen brukes det at Fouriertransformen
\[
\hat{f}(w)=\frac{1}{\sqrt{2\pi}}\int_{-\infty}^{\infty}f(x)e^{-iwx} dx
\]
\begin{oppgave}[5]
Finner Fouriertransformen ved definisjonen over.
\begin{align*}
\hat{f}(w)&=\frac{1}{\sqrt{2\pi}}\int_{-\infty}^{\infty}e^x e^{-iwx} dx \\
&=\frac{1}{\sqrt{2\pi}}\int_{-a}^{a}e^{(1-iw)x} dx \\
&=\frac{1}{\sqrt{2\pi}(1-iw)}\left[e^{(1-iw)x}\right]_{-a}^{a} \\
&=\frac{1}{\sqrt{2\pi}(1-iw)}\left[e^{(1-iw)a}-e^{-(1-iw)a}\right] \\
\end{align*}
\end{oppgave}
\begin{oppgave}[7]
Finner Fouriertransformen.
\begin{align*}
\hat{f}(w)&=\frac{1}{\sqrt{2\pi}}\int_{0}^{a}x e^{-iwx} dx \\
&=\frac{1}{\sqrt{2\pi}}\left[\frac{ix}{w}e^{-iwx}\right]_0^a
	 - \frac{i}{w\sqrt{2\pi}}\int_0^a e^{-iwx} dx \\
&=\frac{ixe^{-iwa}}{w\sqrt{2\pi}}
	 + \frac{1}{w^2\sqrt{2\pi}}\left[e^{-iwx}\right]_0^a \\
&=\frac{iwae^{-iwa}+e^{-iwa}-1}{w^2\sqrt{2\pi}} \\
&=\frac{e^{-iwa}(1+iwa)-1}{w^2\sqrt{2\pi}}
\end{align*}
\end{oppgave}
\begin{oppgave}[9]
Observerer at 
\[
\hat{f}(w)=\frac{1}{\sqrt{2\pi}}\int_{-1}^{1}|x| e^{-iwx} dx
	=\frac{\sqrt{2}}{\sqrt{\pi}}\int_{0}^{1}x e^{-iwx} dx 
\]
Vi har dermed et integral på samme form som i forrige oppgave, men $a=1$. Utnytter resultatet vi allerede har funnet.
\[
\hat{f}(w) = \frac{1}{w^2}\sqrt{\frac{2}{\pi}}(e^{-iw}(1+iw)-1)
	 = \frac{1}{w^2}\sqrt{\frac{2}{\pi}}(\cos(w)-w\sin(w)-1)
\]
Ved siste likhet brukes Eulers formel, $e^{ix} = \cos(x)+i\sin(x)$
\end{oppgave}
\end{document}
